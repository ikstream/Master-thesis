%%%%%%%%%%%%%%%%%%%%%%%%%%%%%%%%%%%%%%%%%%%%%%%%%%%%%%%%%%%%%%%%%%%%%%%%%%
%%%%%%%%%%%%   CAPTER 1   %%%%%%%%%%%%%%%%%%%%%%%%%%%%%%%%%%%%%%%%%%%%%%%%
%%%%%%%%%%%%%%%%%%%%%%%%%%%%%%%%%%%%%%%%%%%%%%%%%%%%%%%%%%%%%%%%%%%%%%%%%%
\chapter{Introduction}
\label{chap:introduction}
%
Developing a multi-platform application is a challenging task. Especially if 
there is no or only sparse telemetry and deployment data. Software needs extended testing to prove it's reliability. But tests reflect only some part of the real word scenarios deployed software faces on a day to day basis. While developers provide sophisticated test suites on some projects, the software might still fail in the end-user's environment. This problem is reflected in message board posts or GitHub issues of annoyed users to fix the issue they encounter.\\

To counter this problem projects may want to gather information, but 
collecting data on user behavior or equipment is a delicate task. On the one hand, there are legal hurdles like the European Unions General Data Protection Regulation (GDPR). On the other hand, is the user's trust in a software project. If a project loses users' trust, it may face a massive loss in user count. This can lead to a drop in revenue for commercial projects and a major setback for any open source project. If the GDPR is violated the project may be significantly damaged, due to considerable fines.\\
Nevertheless, collecting data is important to evolve and improve software. This may be through user surveys, download statistics, or automated collection on systems.
Usage data can help shape a project to attract more users or increases its security and reliability.
It can help identify no longer or intensely used components, which can help to improve maintenance processes.\\
While questionnaires allow a broad range of questions, the number of users participating is usually quite low. While Stack Overflow, the go-to site for questions around programming for many people, has about 50 million visitors per month, only 65.000 developers participated in the 2020 developer survey\cite{noauthor_stack_nodate}.
Especially with Open Source Software, download numbers may be misleading to the actual numbers of users, as the software can typically be compiled by the user. Also, one download may be responsible for multiple installed instances on a various number of different systems, and alternative download possibilities like torrents may further reduce the number of downloads seen on a website.\\

Compared to the two options prior, collecting data through the software, be it operating system or app, provides the most and most accurate results. Depending on the privacy awareness, this option may be counterproductive, as users may shy away from overly nosy apps or operating systems. To improve the quality of a system and satisfy the user, both necessities, the user need for privacy and the developer's urge to get reliable statistics, have to be brought together.\\

This chapter will discuss the problems developers are currently facing in \ref{sec:intro:probstatement} and how
we intend to contribute with this thesis to solve this problem. Furthermore, an 
overview of the thesis structure is provided in \ref{sec:intro:outline}




%%%%%%%%%%%%%%%%%%%%%%%%%%%%%%%%%%%%%
%%%%%%%%%%%%%%%%%%%%%%%%%%%%%%%%%%%%%
%%%%%%%%%%%%   SECTION   %%%%%%%%%%%%
%%%%%%%%%%%%%%%%%%%%%%%%%%%%%%%%%%%%%
%%%%%%%%%%%%%%%%%%%%%%%%%%%%%%%%%%%%%
\section{Problem Statement}
\label{sec:intro:probstatement}
%
In community-based software development projects for multi-platform systems, such as
OpenWrt \cite{brown_welcome_2016}, Tasmota \cite{tasmota_news_nodate}, and similar, the access to information on which and on how many platforms
the software is used is very limited to developers.
Such statistics on the temporal spread of the hardware platforms used by
the end-user (device life-cycle) can improve the software development process.
Likewise, the process of troubleshooting issues, development of patches and features can be prioritized based on:
\begin{itemize}
    \item Which platforms are currently most used and in need of additional testing
    \item Detect end-of-life platforms to remove from the development cycle
    \item Identify systems with software faults
    \item Distribution of software updates
\end{itemize}

Especially software engineering for embedded systems is an intense task due to the wide range of tests.
These include compile, flash, boot, as well as functionality testing for a range of targets. These with higher distributions could be prioritized.
This would allow to roll out, e.g., security updates to as many users as possible in a short time.\\

Additionally, developers of open source projects, like the Internet of Things (IoT) applications, could receive useful
information on the update behavior of their user base. Based on this data, warnings to update the firmware of the
connected device could be issued on the control application of the device.\\
This could lead to an overall increase in network security and hinder the creation of new botnets.\\

\newpage

Collecting hard- and software data on user devices faces challenges. The collected data needs to be anonymized,
so no conclusion about the user can be drawn from the data set. Therefore the collected information should exclude
sensitive data or be anonymized. 
In addition, the user location should not be derivable from the transmitted data or the transmission itself. Simultaneously, the data should be obscured from any eavesdropper during the client-server communication. Even without collecting sensible data, the user's trust in the project is essential. If users do not trust in the collection process, they might stop using a projects software.

Furthermore, a valid database of software and hardware distribution for developers should emerge. In addition,
each device should be recorded and included in the statistics only once.
Thus, the manipulation of the statistics is made more difficult.



%%%%%%%%%%%%%%%%%%%%%%%%%%%%%%%%%%%%%
%%%%%%%%%%%%%%%%%%%%%%%%%%%%%%%%%%%%%
%%%%%%%%%%%%   SECTION   %%%%%%%%%%%%
%%%%%%%%%%%%%%%%%%%%%%%%%%%%%%%%%%%%%
%%%%%%%%%%%%%%%%%%%%%%%%%%%%%%%%%%%%%
\section{Contributions}
\label{sec:intro:contrib}
%

This thesis discusses current systems for data collection and its anonymous transmission. Based on the discussion, we provide a blueprint of how the right to privacy and urge to collect data may be solved.
In addition, an open source implementation will be provided for reference.
This will include data collection, transmission on the client-side, and receiving data and storage on the server-side.

%%%%%%%%%%%%%%%%%%%%%%%%%%%%%%%%%%%%%
%%%%%%%%%%%%%%%%%%%%%%%%%%%%%%%%%%%%%
%%%%%%%%%%%%   SECTION   %%%%%%%%%%%%
%%%%%%%%%%%%%%%%%%%%%%%%%%%%%%%%%%%%%
%%%%%%%%%%%%%%%%%%%%%%%%%%%%%%%%%%%%%
\section{Thesis outline}
\label{sec:intro:outline}

The rest of this thesis is organized as follows. In chapter \ref{chap:related_work} we discuss related work to our research. Chapter \ref{chap:software_design} discusses the decision of the software design process and how we are going to solve the stated problem. In chapter \ref{chap:mmeasurement} solutions for the server-side are presented, and chapter \ref{chap:results} will show the limitations of our software design, as well as tests we conducted. In chapter \ref{chap:conclusion} we conclude this thesis and identify possibilities for future research and improvements.

%\begin{compactitem}
%  \item In chapter 2, we discuss work related to our research
%  \item In chapter 3, the chosen algorithms and medical applications are outlined, as well as the Quality of Service requirements
%  \item In chapter 4, the the hard- and software in described detail.
%  \item In chapter 5, we discuss our measurement methods.
%  %\item In Chapter 5, we present the design, the Linux implementation, the validation and performance evaluation of our controller.
%  \item In chapter 6, we demonstrate our results
%  \item Chapter 7, summarises the contributions and limitations of our systems, and outline several directions for future work.
%\end{compactitem}
