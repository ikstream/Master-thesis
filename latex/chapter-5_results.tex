\chapter{Results}
\label{chap:results}

While the presented solution addresses the need for a privacy-preserving, secure data collection tool, some points could not be addressed. 
Using the structure of DNS, we were able to detach the client's IP from the data, which, in turn, is transported encrypted utilizing asynchronous encryption. 

This chapter provides information on our evaluation steps and discusses the results of our measurement setup described in the last chapter.\\

%%%%%%%%%%%%%%%%%%%%%%%%%%%%%%%%%%%%%
%%%%%%%%%%%%%%%%%%%%%%%%%%%%%%%%%%%%%
%%%%%%%%%%%%   SECTION   %%%%%%%%%%%%
%%%%%%%%%%%%%%%%%%%%%%%%%%%%%%%%%%%%%
%%%%%%%%%%%%%%%%%%%%%%%%%%%%%%%%%%%%%
\section{Limitations of the collection}
\label{sec:measurement:robust}
%

    We looked into ways to validate the collected data. If development is based on the data, adversaries might try to generate emulated devices. We only use physical devices with PCI addresses for the ID generation to counter the emulation. This excludes Linux containers like docker without network card passthrough. The kernel's message buffer is checked if the kernel is loaded in a virtual environment like KVM or Virtualbox. The client checks this using the \textit{dmesg} application. 
    If a virtual environment is detected, the client stops before transmitting any data.\\
    
    Another option to check for validity is to check the integrity of the source code. Therefore a hash or similar can be generated over the existing source code and transmitted to the server.
    If a cryptographic hash function is used, source code modification will show in the hash value transmitted.\\
    This and other options can be leveraged with patching the source code. Especially in open source software, it is easy to manipulate the software. The function that generates the hash of the source code can be pointed to intact source code, while the running code is modified. While it is more complicated, the same is true for closed source software.
    Using a closed source implementation for data collection in an open source project might reduce users' trust in a project and make them stop using it.
    Changing the MAC address of a physical device can be used to manipulate the collection as well. This can be countered on MTD devices by basing the ID generation on the persistent MTD partitions only. For devices without persistent special data partitions, there is currently no workaround in dalec.\\
    
    A user authentication process would be needed to increase the complexity of manipulation. If a malicious actor would need to create one account or token per counterfeit device, the manipulation possibilities could be limited. On the other hand, users with a multitude of legitimate devices would be burdened additionally.\\ 
    As we pointed out in previous chapters, this would reduce the number of participating users, even with privacy-preserving schemes available.\\

    While our system provides privacy, the generated traffic might look malicious.
    In strongly regulated or monitored environments, the DNS requests might trigger data exfiltration alerts. This might lead to the collecting domain ending up on a distributed block list or a drop of all traffic from a client. That may cause issues for other clients in less regulated environments, which solely block DNS requests based on these lists.
    Therefore, clients should be made aware of the data collection layout. The collection domain should ideally state its purpose and be openly documented. Blocking might be prevented that way. A network operator might check on domain names before blocking if the core domain looks unsuspicious.\\
    
    If the DNS resolver is hosted inside the local network of a dalec user instead on a remote location by an ISP or DNS provider, the IP of the user is leaked to the collection server. While this is a possible situation, the collection server would not know the difference. If the incoming IPs are logged, the transmission could be traced back to that user. To prevent IP leakage, users should be made aware of the DNS usage and the ramifications of hosting a DNS resolver in there local network in combination with dalec.\\

    If a project intends to publish data based on PII, all data sets should be checked against the privacy publishing models we presented in section \ref{subsec:related:private_data_analysis}. While we recommend avoiding collecting personally identifiable data, there might be a reason to do so.\\
    There are some open source tools available to ease the privacy-preserving analysis.
    For differential privacy, there are tools developed by IBM \cite{noauthor_ibmdifferential-privacy-library_2021}, Google \cite{noauthor_googledifferential-privacy_2021} or the Harvard University Privacy Tools Project \cite{salil_vadhan_opendp_nodate}. These tools support developers who want to use differential privacy in their applications and analyses.
    Tools for \textit{k-}anonymity and \textit{l-}diversity are available as well. 
    Google provides a solution to calculate values for both algorithms \cite{noauthor_computing_nodate}, \cite{noauthor_computing_nodate-1}. 
    Crowds \cite{mazzone_leo-mazzcrowds_2021} is a Python module that transforms Pandas dataframes with the Optimal Lattice Anonymization algorithm that they satisfy \textit{k-}anonymity. An implementation for \textit{l-}diversity \cite{gong_qiyuangongmondrian_l_diversity_2021} and \textit{k-}anonymity \cite{gong_qiyuangongmondrian_2021} using the Mondrian algorithm for anonymization is available as well.

%
%%%%%%%%%%%%%%%%%%%%%%%%%%%%%%%%%%%%%
%%%%%%%%%%%%%%%%%%%%%%%%%%%%%%%%%%%%%
%%%%%%%%%%%%   SECTION   %%%%%%%%%%%%
%%%%%%%%%%%%%%%%%%%%%%%%%%%%%%%%%%%%%
%%%%%%%%%%%%%%%%%%%%%%%%%%%%%%%%%%%%%
\section{Anonymization Results}
\label{sec:results:anon}
    While the generated IDs use encryption and hashing that has no known feasible attacks at the time of writing, the possibility of reidentification can't be excluded. While the randomness of the data used to generate the ID is sufficient on almost all devices, there are devices with an increased risk.\\
    These are devices with a single network interface card (NIC) and no bootloader or calibration partitions (WOBL) . During the transport, the user data is encrypted. An adversary that monitors a client's traffic could collect the collection query, but without the private key, the collected data is of no use. If an attacker knows the network interface card vendor, there are $2^{24} = 16777216$ hashes to generate, which is a computational feasible task on current hardware. The generated hashes take up around \SIlist{2.2}{\giga\byte} of disk storage.
    If the attacker already obtained the vendor ID, there is only little insight gained, especially if the collected data is stored with a second layer encryption using compute intense algorithms like \textit{b}- or \textit{scrypt}. 
    Then again, if the offender is unaware of the network device manufacturer, the search spaces for MAC generation increase significantly. Even if someone gains access to a full dataset without additional encrypted IDs, the collected data should be too generic to conclude a single device.\\
    For multi-MAC devices, the number of MAC addresses is large enough to be unfeasible to generate all hashes and apply the encryption function. Even if the vendor bits are known, there are $2^{48} \sim 2.81^{14}$ possible addresses.\\
    If the thread model of a user with a single MAC device, without bootloader or calibration data partitions, includes a threat actor that can monitor the user's network traffic and get in possession of the private key from the collection server, we would recommend not to participate. For the regular user, this threat model is unlikely, but for people who participate in high risk political work this may be applicable. These people are probably unlikely to participate. While we classify the risk of reidentifying these devices as low, individual users might have a higher risk level.\\
    Nevertheless, it is a known weakness in our system and will be further investigated.\\
    
    Another feature in modern system that supports our anonymization efforts is load balancing. Internet Service Provider (ISP) and DNS service provider usually try to split the load evenly over their server. Therefore, messages belonging to the client batch may arrive over different resolvers, further reducing the amount of IPs the server receives messages from. While the dalec server can still link each message to the correct client, an attacker would need to monitor more than one DNS server to collect all messages sent between the dalec client and server. If the dalec server would collect the IPs from incoming messages, it could include a wide range of IPs for each client. As the DNS service provider balances the requests between different servers, the physical location of these servers might be widespread, even over multiple countries, as has been encountered during the testing stage.\\
    
    A user, which is especially privacy focus, could still change its preferred DNS server in its router or device settings to a service that is not bound to an ISP. This reduces the risk that an attacker or a collection server might determine a user's country.\\
    To enhance privacy further, technologies like DoH, ODoH, and DNSSec can be used. These can reduce the risk of an ISP or an intruder of the ISP's network listening in on the connection. 




%%%%%%%%%%%%%%%%%%%%%%%%%%%%%%%%%%%%%
%%%%%%%%%%%%%%%%%%%%%%%%%%%%%%%%%%%%%
%%%%%%%%%%%%   SECTION   %%%%%%%%%%%%
%%%%%%%%%%%%%%%%%%%%%%%%%%%%%%%%%%%%%
%%%%%%%%%%%%%%%%%%%%%%%%%%%%%%%%%%%%%
\section{Performance evaluation}
\label{sec:results:telemetry}
%
    
    To measure the load on a network, we generated 25 x86$\_$64 virtual machines (VM) with KVM (see listing \ref{lst:vm-spec}). We disabled dalec's virtualization checks to run the software on these machines. We configured dalec to transmit the data at the same time. The server was able to sort all messages to the correct client IDs and decrypt the data afterwards. The server performance can be increased further by adding more processing threads in each step. The implementation of interprocess communication allows scalability. The bottleneck could be handling incoming data, which could be solved with multiple load-balanced server instances. Sending all client requests to the same server would depend on the load balancer's configuration.\\
    \begin{lstlisting}[language=json, caption=virtual machine specifications, label=lst:vm-spec]
    RAM: 128 MB
    CPU: 1
    Network interface count: 2
    Operating System: OpenWrt
    \end{lstlisting}
    The structure of a transmitted message can be seen in listing \ref{lst:message}. The query consists of three labels that contain data fragments, followed by the combination of ID (\textit{EPfGiM9IJpDHQQMaItynBRlJpusUP0KA}), current message number (7) and total number of messages (8).     The next field contains the \textit{owrt} subdomain. The domain \textit{sviks} is followed by the top level domain (TLD) \textit{de}. The root "." was omitted.
    As can be seen. the total transmission data was split over eight messages.
    They are sent out as eight DNS queries, with one additional query in the beginning for the public key of the dalec server.
    Data transmission one to seven had a size of 296 bytes on the wire with a name length of 236 bytes and data transmission eight had a size of 221 bytes with a name length of 161 bytes. Eight messages of  25 devices sum up to \SIlist{55.825}{\kilo\byte} of network load. Dalec is configured to run every 4 hours to make up for transmission errors. 
    This sums up to a total of \SIlist{334.95}{\kilo\byte} per day for 25 clients. 
    A single client produces around \SIlist{14}{\kilo\byte} of outgoing traffic per day. If the DNS resolver returns a failed response, the number of message increases and consumed data increases.

    \begin{lstlisting}[language=json, caption=Layout of dalec message, label=lst:message]
634c716a66664e61575a5936664a704d51687a4a73454f53354f69314e524.162506a310a636136794752496d36794c5352626a6a464d765532636f4835.7258694279492b6b714c65507138426667596a31382b597a756d544267643.EPfGiM9IJpDHQQMaItynBRlJpusUP0KA-7-8.owrt.sviks.de
    \end{lstlisting}    
    
    To test the reidentification resistance on WOBL devices, we generated another virtual machine, as shown in listing \ref{lst:vm-spec}, but with only one network interface. 
    We generated all possible MAC addresses of the virtualized network interface with the vendor prefix \textit{52:54:00}, which took around 30 seconds. We used the tool \textit{exrex} \cite{tauber_asciimooexrex_2021} for the generation as seen in listing \ref{lst:exrex}. 
    
    \begin{lstlisting}[language=json, caption=MAC generation with exrex, label=lst:exrex]
    exrex '525400[a-f0-9]{6}'
    \end{lstlisting}    
    
    We hashed the generated MAC addresses as done in dalec. The command is shown in listing \ref{lst:hash}.
    
    \begin{lstlisting}[language=json, caption=MAC hashing with OpenSSL, label=lst:hash]
openssl dgst -sha3-512
    \end{lstlisting}
    
    The generated hashes were saved to a file. Hashing all MAC addresses took around 13.5 hours with a single instance of \textit{OpenSSL}.
    We utilized an AMD Ryzen 5 3600X for testing. The core speed was fixed to \SIlist{3.8}{\giga\hertz}. The PC is equipped with \SIlist{64}{\giga\byte} of RAM.
    Multiple instances and splitting up the list of MAC addresses can speed up the hashing process.
    We split up the output file into ten similar sized chunks to encrypt the generated hashes. Each chunk was processed as shown in \ref{lst:encryption}.
    
    \begin{lstlisting}[language=json, caption=ID encryption and truncation, label=lst:encryption]
    openssl enc -aes-256-cbc -md sha512 -pbkdf2 -iter 100000 \                          
        -k "$key" -S "$salt" -base64 | \                                      
        tr -d '\n' | \                                                        
        sed 's;[^a-zA-z0-9];;g' | \
        tail -c 32
    \end{lstlisting}
    
\newpage
    
    Encrypting and writing a single hash to file takes around \SIlist{0.25}{\second} on the test system. Processing all hashes with a single instance of \textit{OpenSSL} takes over 1165 hours. Splitting it over ten simultaneous operations reduces the time to 116.5 hours.
    An adversary without the vendor prefix would have to do this for a range of prefixes, increasing the processing time significantly to a computationally infeasible extend.\\
    
    \textit{Dalec} has been tested on different OpenWrt releases and devices. While Snapshots and 19.07.7 worked without additional changes, the 17.01.7 and 18.06.9 release would not run it out of the box. For 18.06.9 manual updates of \textit{openssl-util, libopenssl} and \textit{libopenssl-conf} were needed. 
    In addition to the aforementioned packages the 17.01.7 release required \textit{drill} and \textit{libldns} to be updated manually.\\
    Table \ref{tbl:dev_test} lists the devices with which \textit{dalec} was tested with the respective OpenWrt version. The Edimax RA21S and Raspberry 4 are supported in Snapshots only at the time of writing. Support for the TP-Link CPE210 was added 18.06.0 but could not be downgraded, as it is in productive use and relies on software added in 19.07.0.
    
   \begin{table}
    \centering
    \caption{List of tested devices with release version}
    \label{tbl:dev_test}
    \begin{tabular}{|l|c|c|c|c|} 
        \hline
        Device & \multicolumn{4}{c|}{Release} \\ 
        \hline
               & \multicolumn{1}{l|}{Snapshot} & \multicolumn{1}{l|}{19.07.7} & \multicolumn{1}{l|}{18.06.9} & \multicolumn{1}{l|}{17.01.7}  \\ 
        \hline
        Raspberry Pi 2 B 1.2 & x & x & x & x \\ 
        \hline
        Raspberry Pi 3 B & x & x & x & x \\ 
        \hline
        Raspberry Pi 4 400 & x & & & \\ 
        \hline
        Alix board AMD Geode LX800 based & x & x & x & x \\ 
        \hline
        PC Engines APU 1D4 & x & x & x & x \\ 
        \hline
        TP-Link CPE210 v2 & x & x & \multicolumn{1}{l|}{} & \multicolumn{1}{l|}{} \\ 
        \hline
        Archer C7 AC1750 v2.1 & x & x & x & x \\ 
        \hline
        Edimax RA21S & x & \multicolumn{1}{l|}{} & \multicolumn{1}{l|}{} & \multicolumn{1}{l|}{} \\
        \hline
    \end{tabular}
\end{table}
%%%%%%%%%%%%%%%%%%%%%%%%%%%%%%%%%%%%%
%%%%%%%%%%%%%%%%%%%%%%%%%%%%%%%%%%%%%
%%%%%%%%%%%%   SECTION   %%%%%%%%%%%%
%%%%%%%%%%%%%%%%%%%%%%%%%%%%%%%%%%%%%
%%%%%%%%%%%%%%%%%%%%%%%%%%%%%%%%%%%%%
\section{Summary}
In this chapter, we demonstrated the limitations of the system. We debated especially on the validity of the collected data, as these is probably the hardest limitation to overcome.
We also discussed the security of the generated IDs during the transport and storage phase. We gave an overview of the complexity to reidentify a WOBL device. Especially for devices with only one physical interface, this is critical. We also discussed under which circumstances we would recommend not to participate in the data collection.
In the next chapter will conclude and summarize this thesis.

