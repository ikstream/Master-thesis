\addcontentsline{toc}{chapter}{Zusammenfassung}
\chapter*{Zusammenfassung}
\label{chap:abstract_de}

Vor allem Open Source Projekte für mehrere Plattformen, wie OpenWrt oder Tasmota haben in der Regel sehr wenige Informationen über die Anzahl und Art der Plattformen auf denen ihre Software zum Einsatz kommt. Solche gesammelte Daten kann dabei helfen den Entwicklungsprozess auf häufig genutzten Plattformen zu beschleunigen und bisher unbekannte Probleme auf weniger genutzten aufzuzeigen. Die Datensammlung ist jedoch eine delikate Angelegenheit, da hierbei sehr genau darauf geachtet werden muss, welche Daten erhoben werden um die Privatsphäre der Nutzer zu bewahren. Sollten unter den gesammelten personenbezogene Daten sein, könnten einzelne Nutzer ausfindig gemacht werden und gezielt kompromittiert werden.\\

In dieser Arbeit werden Richtlinien zum erheben der Daten besprochen. Weiterhin werden Empfehlungen gegeben, wie der Transport verschlüsselt und anonymisiert über DNS realisiert werden kann. Um einzelne Geräte unterscheiden zu können bedarf es einer ID. Wie diese generiert werden kann, wird ebenso diskutiert, wie die eigentliche Auswahl der Daten. Weiterführend wird über die Umsetzung des Empfangs-Servers gesprochen und wie dieser abzusichern ist. Darüber hinaus wird auf allgemeine Empfehlung zur sicheren und nachvollziehbaren Verwaltung von Daten eingegangen.\\

Weiterhin wird eine Referenzimplementierung (dalec), sowohl auf Server, als auch auf Client Seite, vorgestellt. Hierbei werden getroffene Design-Entscheidungen erläutert und die Sicherheit der generierten ID, insbesondere bei Geräten mit geringer Entropie als Grundlage des Identifiers, betrachtet

%

