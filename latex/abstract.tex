\addcontentsline{toc}{chapter}{Abstract}
\chapter*{Abstract}
\label{chap:abstract}

%
Especially open source projects for multiple platforms, like OpenWrt or Tasmota usually have very little information about the number and type of platforms their software is used on. Such collected data can help to speed up the development process on commonly used platforms and to identify previously unknown problems on less used ones. However, data collection is a delicate matter, as great care must be taken about what data is collected in order to preserve user privacy. If the data collected includes personal data, individual users could be identified and specifically compromised.\\

In this paper, guidelines for collecting the data are discussed. Furthermore, recommendations are given on how the transport can be realized encrypted and anonymized via DNS. An ID is required to distinguish between individual devices. We will present the generation of such IDs and how the actual data is selected. In addition, the implementation and security measurements of the receiving server are discussed. Further on, general recommendations for the secure and traceable management of data are addressed.\\

Furthermore, a reference implementation (dalec), both on server and client side, is presented. Design decisions are explained and the security of the generated ID is considered, especially for devices with low entropy as the basis of the identifier.