\documentclass[
% papersize
a4paper,
%fond size
%12pt,
11pt,
%page print style
twoside,
%oneside,
openright]{book}

%for PSTRicks
\usepackage[pdf]{pstricks}
\usepackage[off]{auto-pst-pdf}

%graphics included tikz & pfg
\usepackage{graphicx}
\usepackage{epsfig,epic,eepic}
\usepackage{psfrag}
\usepackage[USenglish,german]{babel}
\usepackage{color,pstricks}
\usepackage{tikz}
\usepackage{ctable}
\usepackage[utf8]{inputenc}
\usepackage{wrapfig}

\usepackage[titletoc]{appendix}
\usepackage{parskip}
\usepackage{diagbox}

% Fonts
%\usepackage{lmodern}
%from Ruben \usepackage[T1]{fontenc}
\usepackage{times}
%\usepackage[scaled]{helvet}
%\usepackage[sc]{mathpazo} % math fonts for Palatino, see
                          % http://www.tug.dk/FontCatalogue/palatino/
%\usepackage{palatino}

% Math packages
\usepackage{amsmath,amssymb,amsfonts,amsthm}
\usepackage{url}
\usepackage[ruled,noline]{algorithm2e}
% More compact lists
\usepackage{paralist}
% Fullpage
\usepackage[in,headings]{fullpage}
% But then adapt spacing
\usepackage{setspace}
\setstretch{1.3}

%for nice tables
\usepackage{multirow}
%\usepackage{booktabs}

% For SI Unit conform output
\usepackage[binary-units = true]{siunitx}

%for nice item lists
\usepackage{paralist}

%usepackage comment for block comments
\usepackage{comment}

%nice symbols to draw arrow (yes) and -(no)
\usepackage{pifont}
\newcommand{\yes}{\checkmark}
\newcommand{\no}{\textendash}

%sub-picture numbering
\usepackage{caption}
\usepackage{subcaption}
%\usepackage{subfigure}

\usepackage[right]{eurosym}


%annotate TODO stuff
\usepackage[colorinlistoftodos, textwidth=4cm, shadow]{todonotes}
\usepackage{lscape}
\usepackage{epstopdf}
%pseudocode package
\usepackage{pseudocode}

%for pdf includes
\usepackage[final]{pdfpages}

% allow captions for listings
\usepackage{newfloat}
\DeclareFloatingEnvironment[placement={!ht},name=List]{mylist}

%to insert some clickable links
\usepackage{hyperref}
\definecolor{darkblue}{rgb}{0,0,.5}
\definecolor{black}{rgb}{0,0,0}
\hypersetup{colorlinks=true, breaklinks=true, citecolor=black, linkcolor=black, menucolor=black, urlcolor=black}

% Allow code blocks
\usepackage{listings}

% Add syntax highlighting for json
\definecolor{eclipseStrings}{RGB}{42,0.0,255}
\definecolor{eclipseKeywords}{RGB}{127,0,85}
\definecolor{bground}{HTML}{f2f2f2}
\colorlet{numb}{magenta!60!black}

\lstdefinelanguage{json}{
    basicstyle=\normalfont\ttfamily,
    commentstyle=\color{eclipseStrings}, % style of comment
    stringstyle=\color{eclipseKeywords}, % style of strings
    numbers=left,
    numberstyle=\scriptsize,
    stepnumber=1,
    numbersep=8pt,
    showstringspaces=false,
    breaklines=true,
    frame=lines,
    backgroundcolor=\color{bground}, %only if you like
    string=[s]{"}{"},
    comment=[l]{:\ "},
    morecomment=[l]{:"},
    literate=
        *{0}{{{\color{numb}0}}}{1}
         {1}{{{\color{numb}1}}}{1}
         {2}{{{\color{numb}2}}}{1}
         {3}{{{\color{numb}3}}}{1}
         {4}{{{\color{numb}4}}}{1}
         {5}{{{\color{numb}5}}}{1}
         {6}{{{\color{numb}6}}}{1}
         {7}{{{\color{numb}7}}}{1}
         {8}{{{\color{numb}8}}}{1}
         {9}{{{\color{numb}9}}}{1}
}

% \setlength{\oddsidemargin}{-0.3cm}
% \setlength{\evensidemargin}{-0.3cm}
% \setlength{\textwidth}{16cm}
% \setlength{\topmargin}{-0.8cm}
% \setlength{\textheight}{23.0cm}

% use some common definitions
\newcommand{\pref}[1]{Part~\ref{#1}}
\newcommand{\cref}[1]{Chapter~\ref{#1}}
\newcommand{\sref}[1]{Section~\ref{#1}}
\newcommand{\aref}[1]{Appendix~\ref{#1}}
\newcommand{\fref}[1]{Figure~\ref{#1}}
\newcommand{\tref}[1]{Table~\ref{#1}}
\newcommand{\eref}[1]{(\ref{#1})}

% More convenient sometimes
\newcommand{\ie}{i.\@e.\@ }
\newcommand{\eg}{e.\@g.\@ }

\newcommand{\us}{\,$\mu$s\xspace}

%add empty page macro
\newcommand{\blankpage}{
\newpage
\thispagestyle{empty}
\mbox{}
\newpage
}

% Some math definitions
\input{local_tex_packages/mathdef}

% Style headings + headers and footers
\input{local_tex_packages/style}

% Removing master, slows down the compilation
\includeonly{
	title,
	abstract,
	abstract_de,
	acknowledgments,
	paper-collaborators,
	eid,
	chapter-1_introduction,
	chapter-2_related_work,
	chapter-3_software_design,
	chapter-4_measurement_tools,
	chapter-5_results,
	chapter-6_conclusion,
	appendix,
}


% BEGINN of COCUMENT
\begin{document}

\setlength{\headheight}{13.6pt}

%TITLE PAGE
\pagestyle{empty}
% Titeseite

\begin{titlepage}
  \begin{center}
	  
    %\mbox{}

    %\vspace{1cm}

	\centering
	\includegraphics[width=15cm]{figures/Q03_HTW_Berlin_Logo_quer_pos_FARBIG_RGB.eps}\\
	\vspace{0.8em}
	\LARGE 
    \sffamily \LARGE \Huge{\textbf{Community friendly tracking of Hardware Target Distributions and Software releases}}

	\vspace{1cm}
	
	\large Master Thesis

    \vspace{1cm}

    \normalsize Informations- und Komminkationstechnik\\
    \normalsize Fachbereich I -- Ingenieurwissenschaften - Energie und Information\\
    
    \vspace{1cm}
    
    \large \textbf{Submitted By:}\\
    \Large Stefan Venz


  %  von der Fakult�t IV - Elektrotechnik und Informatik der Technischen Universit�t Berlin



    %\vspace{.1cm}
    %\normalsize der Hochschule für Technik und Wirtschaft Berlin\\
    %\vspace{.1cm}
    %\normalsize zur Erlangung des akademischen Grades\\
    %\vspace{.1cm}
    %\large \textsc{Bachelor of Engineering (B. Eng.)}\\
    %\vspace{.1cm}
    %\normalsize genemigte Bachelorarbeit\\

    \vspace{1cm}

    \large \textbf{Submitted To:}\\
        \vspace{2mm}
		\begin{tabular}{rl}
			& Prof. Dr.-Ing. Thomas Scheffler, HTW Berlin\\
			& Prof. Dr.-Ing. Thomas Hühn, Hochschule Nordhausen\\
			%ToDo: is our adress needed here ?
		\end{tabular}\\

    \vspace{2cm}

    \large Berlin, April 2021\\

  \end{center}
\end{titlepage}


\selectlanguage{USenglish}

%ABSTRACT_EN
\frontmatter \pagestyle{plain}
\setcounter{page}{1}
\addcontentsline{toc}{chapter}{Abstract}
\chapter*{Abstract}
\label{chap:abstract}

%
Especially open source projects for multiple platforms, like OpenWrt or Tasmota usually have very little information about the number and type of platforms their software is used on. Such collected data can help to speed up the development process on commonly used platforms and to identify previously unknown problems on less used ones. However, data collection is a delicate matter, as great care must be taken about what data is collected in order to preserve user privacy. If the data collected includes personal data, individual users could be identified and specifically compromised.\\

In this paper, guidelines for collecting the data are discussed. Furthermore, recommendations are given on how the transport can be realized encrypted and anonymized via DNS. An ID is required to distinguish between individual devices. We will present the generation of such IDs and how the actual data is selected. In addition, the implementation and security measurements of the receiving server are discussed. Further on, general recommendations for the secure and traceable management of data are addressed.\\

Furthermore, a reference implementation (dalec), both on server and client side, is presented. Design decisions are explained and the security of the generated ID is considered, especially for devices with low entropy as the basis of the identifier.
\blankpage
%ABSTRACT_DE

 \setcounter{page}{3}
%\begin{otherlanguage*}{german}
%\setcounter{page}{}
\addcontentsline{toc}{chapter}{Abstract}
\chapter*{Zusammenfassung}
\label{chap:abstract_de}

%


\blankpage
 %   \addcontentsline{toc}{section}{Zusammenfassung}
%\end{otherlanguage*}

%ACKNOWLAGEMENTS
%\include{acknowledgments}

%PUBLICATIONS & COLLABORATORS
%\include{paper-collaborators}

%EIDESSTATTLICHE ERKLAERUNG
\includepdf{latex/eid_signed}


%INSERT EMPTY PAGE
\blankpage

%TOC PAGE
 %\setcounter{page}{5}
\tableofcontents

%ABBILDUNGSVERZEICHNIS
%\setcounter{page}{6}

%\renewcommand\thepage{\romannumeral\numexpr\value{page}+1\relax}
%\setcounter{page}{6}
\cleardoublepage
\addcontentsline{toc}{chapter}{List of Figures}
\listoffigures

%TABELLENVERZEICHNIS
%\addcontentsline{toc}{chapter}{List of Tables}
%\listoftables
%\setcounter{page}{7}

\blankpage
%MAIN CONTENT
\mainmatter \pagestyle{fancy}
%\pagestyle{headings}
%%%%%%%%%%%%%%%%%%%%%%%%%%%%%%%%%%%%%%%%%%%%%%%%%%%%%%%%%%%%%%%%%%%%%%%%%%
%%%%%%%%%%%%   CAPTER 1   %%%%%%%%%%%%%%%%%%%%%%%%%%%%%%%%%%%%%%%%%%%%%%%%
%%%%%%%%%%%%%%%%%%%%%%%%%%%%%%%%%%%%%%%%%%%%%%%%%%%%%%%%%%%%%%%%%%%%%%%%%%
\chapter{Introduction}
\label{chap:introduction}
%
Developing a multi-platform application is a challenging task. Especially if 
there is no or only sparse telemetry and deployment data. Software needs extended testing to prove it's reliability. But tests reflect only some part of the real word scenarios deployed software faces on a day to day basis. While developers provide sophisticated test suites on some projects, the software might still fail in the end-user's environment. This problem is reflected in message board posts or GitHub issues of annoyed users to fix the issue they encounter.\\

Collecting data on user behavior or equipment is a delicate task. On the one hand, there are legal hurdles like the European Unions General Data Protection Regulation (GDPR). On the other hand, is the user's trust in a software project. If a project loses users' trust, it may face a massive loss in user count. This can lead to a drop in revenue for commercial projects and a major setback for any open source project. If the GDPR is violated the project may be significantly damaged, due to considerable fines.\\
Nevertheless, collecting data is important to evolve and improve software. This may be through user surveys, download statistics, or automated collection on systems.
Usage data can help shape a project that attracts more users or increases its security and reliability.
It can help identify no longer or intensely used components, which can help to improve maintenance processes.\\
While questionnaires allow a broad range of questions, the number of users participating is usually quite low. While Stack Overflow, the go-to site for questions around programming for many people, has about 50 million visitors per month, only 65.000 developers participated in the 2020 developer survey\cite{noauthor_stack_nodate}.
Especially with Open Source Software, download numbers may be misleading to the actual numbers of users, as the software can typically be compiled by the user. Also, one download may be responsible for multiple installed instances on a various number of different systems, and alternative download possibilities like torrents may further reduce the number of downloads seen on a website.\\

Compared to the two options prior, collecting data through the software, be it operating system or app, provides the most and most accurate results. Depending on the privacy awareness, this option may be counterproductive, as users may shy away from overly nosy apps or operating systems. To improve the quality of a system and satisfy the user, both necessities, the user need for privacy and the developer's urge to get reliable statistics, have to be brought together.\\

This chapter will discuss the problems developers are currently facing in \ref{sec:intro:probstatement} and how
we intend to contribute with this thesis to solve this problem. Furthermore, an 
overview of the thesis structure is provided in \ref{sec:intro:outline}




%%%%%%%%%%%%%%%%%%%%%%%%%%%%%%%%%%%%%
%%%%%%%%%%%%%%%%%%%%%%%%%%%%%%%%%%%%%
%%%%%%%%%%%%   SECTION   %%%%%%%%%%%%
%%%%%%%%%%%%%%%%%%%%%%%%%%%%%%%%%%%%%
%%%%%%%%%%%%%%%%%%%%%%%%%%%%%%%%%%%%%
\section{Problem Statement}
\label{sec:intro:probstatement}
%
In community-based software development projects for multi-platform systems, such as
OpenWrt, Tasmota, and similar, the access to information on which and on how many platforms
the software is used is very limited to developers.
Such statistics on the temporal spread of the hardware platforms used by
the end-user (device life-cycle) can improve the software development process.
Likewise, the process of troubleshooting issues, development of patches and features can be prioritized based on:
\begin{itemize}
    \item Which platforms are currently most used and in need of additional testing
    \item Detect end-of-life platforms to remove from the development cycle
    \item Identify systems with software faults
    \item Distribution of software updates
\end{itemize}

Software engineering for embedded systems is an intense task due to the wide range of tests.
These include compile, flash, boot, as well as functionality testing for a range of targets. These with higher distributions could be prioritized.
This would allow to roll out, e.g., security updates to as many users as possible in a short time.\\

Additionally, developers of open source projects, like the Internet of Things (IoT) applications, could receive useful
information on the update behavior of their user base. Based on this data, warnings to update the firmware of the
connected device could be issued on the control application of the device.\\
This could lead to an overall increase in network security and hinder the creation of new botnets.\\

Collecting hard- and software data on user devices is a delicate action. The collected data needs to be anonymized,
so no conclusion about the user can be drawn from the data set. Therefore the collected information should exclude
sensitive data or should be anonymized. 
In addition, the user location should not be derivable from the transmitted data or the transmission itself. Simultaneously, the data should be obscured from any eavesdropper during the client-server communication. Even without collecting sensible data, the user's trust in the project is essential. If users do not trust in the collection process, they might stop using a projects software.

Furthermore, a valid database of software and hardware distribution for developers should emerge. In addition,
each device should be recorded and included in the statistics only once.
Thus the manipulation of the statistics is made more difficult.



%%%%%%%%%%%%%%%%%%%%%%%%%%%%%%%%%%%%%
%%%%%%%%%%%%%%%%%%%%%%%%%%%%%%%%%%%%%
%%%%%%%%%%%%   SECTION   %%%%%%%%%%%%
%%%%%%%%%%%%%%%%%%%%%%%%%%%%%%%%%%%%%
%%%%%%%%%%%%%%%%%%%%%%%%%%%%%%%%%%%%%
\section{Contributions}
\label{sec:intro:contrib}
%

This thesis discusses current systems for data collection and its anonymous transmission. Based on the discussion, we provide a blueprint of how the right to privacy and urge to collect data may be solved.
In addition, an open source implementation will be provided for reference.
This will include data collection, transmission on the client-side, and receiving data and storage on the server-side.

%%%%%%%%%%%%%%%%%%%%%%%%%%%%%%%%%%%%%
%%%%%%%%%%%%%%%%%%%%%%%%%%%%%%%%%%%%%
%%%%%%%%%%%%   SECTION   %%%%%%%%%%%%
%%%%%%%%%%%%%%%%%%%%%%%%%%%%%%%%%%%%%
%%%%%%%%%%%%%%%%%%%%%%%%%%%%%%%%%%%%%
\section{Thesis outline}
\label{sec:intro:outline}

The rest of this thesis is organized as follows. In chapter \ref{chap:related_work} we discuss related work to our research. Chapter \ref{chap:software_design} discusses the decision of the software design process and how we are going to solve the stated problem. In chapter \ref{chap:mmeasurement} solutions for the server-side are presented, and chapter \ref{chap:results} will show the limitations of our software design, as well as tests we conducted. In chapter \ref{chap:conclusion} we conclude this thesis and identify possibilities for future research and improvements.

%\begin{compactitem}
%  \item In chapter 2, we discuss work related to our research
%  \item In chapter 3, the chosen algorithms and medical applications are outlined, as well as the Quality of Service requirements
%  \item In chapter 4, the the hard- and software in described detail.
%  \item In chapter 5, we discuss our measurement methods.
%  %\item In Chapter 5, we present the design, the Linux implementation, the validation and performance evaluation of our controller.
%  \item In chapter 6, we demonstrate our results
%  \item Chapter 7, summarises the contributions and limitations of our systems, and outline several directions for future work.
%\end{compactitem}

%%%%%%%%%%%%%%%%%%%%%%%%%%%%%%%%%%%%%%%%%%%%%%%%%%%%%%%%%%%%%%%%%%%%%%%%%%
%%%%%%%%%%%%   CAPTER 2   %%%%%%%%%%%%%%%%%%%%%%%%%%%%%%%%%%%%%%%%%%%%%%%%
%%%%%%%%%%%%%%%%%%%%%%%%%%%%%%%%%%%%%%%%%%%%%%%%%%%%%%%%%%%%%%%%%%%%%%%%%%
\chapter{Background and Related Work}
\label{chap:related_work}


%%%%%%%%%%%%%%%%%%%%%%%%%%%%%%%%%%%%%
%%%%%%%%%%%%%%%%%%%%%%%%%%%%%%%%%%%%%
%%%%%%%%%%%%   SECTION   %%%%%%%%%%%%
%%%%%%%%%%%%%%%%%%%%%%%%%%%%%%%%%%%%%
%%%%%%%%%%%%%%%%%%%%%%%%%%%%%%%%%%%%%
\section{Related Software}
\label{sec:related_work:related_sw}
%


%%%%%%%%%%%%%%%%%%%%%%%%%%%%%%%%%%%%%
%%%%%%%%%%%%%%%%%%%%%%%%%%%%%%%%%%%%%
%%%%%%%%%%%%   SECTION   %%%%%%%%%%%%
%%%%%%%%%%%%%%%%%%%%%%%%%%%%%%%%%%%%%
%%%%%%%%%%%%%%%%%%%%%%%%%%%%%%%%%%%%%
\section{Data Anonymization}
\label{sec:related_work:data_aononymization}
%

%%%%%%%%%%%%%%%%%%%%%%%%%%%%%%%%%%%%%
%%%%%%%%%%%%%%%%%%%%%%%%%%%%%%%%%%%%%
%%%%%%%%%%%%   SECTION   %%%%%%%%%%%%
%%%%%%%%%%%%%%%%%%%%%%%%%%%%%%%%%%%%%
%%%%%%%%%%%%%%%%%%%%%%%%%%%%%%%%%%%%%
\section{Data Transmission}
\label{sec:related_work:data_transmission}
%

%%%%%%%%%%%%%%%%%%%%%%%%%%%%%%%%%%%%%
%%%%%%%%%%%%%%%%%%%%%%%%%%%%%%%%%%%%%
%%%%%%%%%%%%   SECTION   %%%%%%%%%%%%
%%%%%%%%%%%%%%%%%%%%%%%%%%%%%%%%%%%%%
%%%%%%%%%%%%%%%%%%%%%%%%%%%%%%%%%%%%%
\section{Summary}
In this chapter we summarised previous work on related topics. This allows us to narrow our research.
Therefor we concentrate our research on <technology> as we want to achieve.
%



%%%%%%%%%%%%%%%%%%%%%%%%%%%%%%%%%%%%%%%%%%%%%%%%%%%%%%%%%%%%%%%%%%%%%%%%%%
%%%%%%%%%%%%   CAPTER 3   %%%%%%%%%%%%%%%%%%%%%%%%%%%%%%%%%%%%%%%%%%%%%%%%
%%%%%%%%%%%%%%%%%%%%%%%%%%%%%%%%%%%%%%%%%%%%%%%%%%%%%%%%%%%%%%%%%%%%%%%%%%
\chapter{Software Design}
\label{chap:software_design}



%%%%%%%%%%%%%%%%%%%%%%%%%%%%%%%%%%%%%
%%%%%%%%%%%%%%%%%%%%%%%%%%%%%%%%%%%%%
%%%%%%%%%%%%   SECTION   %%%%%%%%%%%%
%%%%%%%%%%%%%%%%%%%%%%%%%%%%%%%%%%%%%
%%%%%%%%%%%%%%%%%%%%%%%%%%%%%%%%%%%%%
\section{Parameter}
\label{sec:measurement:parameter}

%

\newpage


%%%%%%%%%%%%%%%%%%%%%%%%%%%%%%%%%%%%%
%%%%%%%%%%%%%%%%%%%%%%%%%%%%%%%%%%%%%
%%%%%%%%%%%%   SECTION   %%%%%%%%%%%%
%%%%%%%%%%%%%%%%%%%%%%%%%%%%%%%%%%%%%
%%%%%%%%%%%%%%%%%%%%%%%%%%%%%%%%%%%%%
\section{Data Collection}
\label{sec:software_design:data_collection}

 %
    %%%%%%%%%%%%%%%%%%%%%%%%%%%%%%%%%%%%%
    %%%%%%%%%%%% Subsection %%%%%%%%%%%%%
    %%%%%%%%%%%%%%%%%%%%%%%%%%%%%%%%%%%%%
    %
    \subsection{Data selection}
        \label{subsec:software_design:selection}
        We decided to restrict the collected data to non-PII. This will allow us to avoid GDPRs regulation. 
        In addition we don't have to take care of anonymization, as the data does not allow any conclusions 
        on the user.

    \subsection{ID generation}
        \label{subsec:software_design:id}
        To identify devices on a reliable and reproducible basis the ID generation is based on hardware information available and is therefore persistent through a reboot, a re-flash or an update.\\
        We recommend the use of the available MAC-addresses and combine them into one string which is then hashed with the a strong cryptographic hashing function.
        This function should come from the Secure Hash Algorithm 2 (SHA-2) family. While SHA512 is the stronger function and is recommended, it is not available on all devices.\\
        A single MAC consists of 48 bit, from which 24 bit are a vendor specific and equal over all devices from this vendor. The other 24 bit are a unique identifier for a given network interface. As most internet connected devices come with pre-installed/soldered network interfaces, this could lead to brute force attempts to regenerate the MAC-address from a hash, especially on hardware with only one physical network interface. Therefore the generated hash needs to be enhanced.\\
        This enforcement can be achieved with a key derivation function (KDF). These functions have two usue cases. On the one hand they are used for password hashes in modern operating systems protecting the users password against easy access\cite{percival_stronger_nodate}. 
        The other use case is the derivation of a key based on a token and another key\cite{camenisch_privacy_2011}. Attacking a KDF in itself is not feasible. An attacker would need to iterate over a range of passwords or regular expressions and apply the KDF to them\cite{percival_stronger_nodate}.
        One example of such a derivation function is Password-Based Key Derivation Function 2 (PBKDF2), in which applies a pseudorandom function to a given password and salt for a number of times. This reduces it's vulnerability against brute force attacks\cite{kaliski_bkaliskirsasecuritycom_pkcs_2000}. 
        Computation time increases significantly for strong passwords and salts, making rainbow table attacks less feasible. 
        To keep IDs reproducible we need to derive our password and salt from data provided by the device, which wont alter during a reinstall of a system. For devices with non volatile flash memory the Memory Technology Devices (MTD) are usually devices with solid state file systems\cite{giometti_mtd_2017}. These partitions can contain configuration information for wireless devices. This contains device-unique data which is ideal to use as key and/or salt.
        To further strengthen the ID against brute forcing and decreasing the amount of character use in a DNS query, the generated ID should be reduced to fewer bytes.\\
        This may lead to collisions
        MTD \cite{woodhouse_memory_nodate}
\newpage



%%%%%%%%%%%%%%%%%%%%%%%%%%%%%%%%%%%%%
%%%%%%%%%%%%%%%%%%%%%%%%%%%%%%%%%%%%%
%%%%%%%%%%%%   SECTION   %%%%%%%%%%%%
%%%%%%%%%%%%%%%%%%%%%%%%%%%%%%%%%%%%%
%%%%%%%%%%%%%%%%%%%%%%%%%%%%%%%%%%%%%
\section{Data Transmission}
\label{sec:software_design:tx}
%
     %
    %%%%%%%%%%%%%%%%%%%%%%%%%%%%%%%%%%%%%
    %%%%%%%%%%%% Subsection %%%%%%%%%%%%%
    %%%%%%%%%%%%%%%%%%%%%%%%%%%%%%%%%%%%%
    %
    \subsection{Data Encryption}
        \label{subsec:software_design:encryption}

     %
    %%%%%%%%%%%%%%%%%%%%%%%%%%%%%%%%%%%%%
    %%%%%%%%%%%% Subsection %%%%%%%%%%%%%
    %%%%%%%%%%%%%%%%%%%%%%%%%%%%%%%%%%%%%
    %
    \subsection{Fitting data to DNS request}
        \label{subsec:software_design:fitting}
\newpage


%%%%%%%%%%%%%%%%%%%%%%%%%%%%%%%%%%%%%
%%%%%%%%%%%%%%%%%%%%%%%%%%%%%%%%%%%%%
%%%%%%%%%%%%   SECTION   %%%%%%%%%%%%
%%%%%%%%%%%%%%%%%%%%%%%%%%%%%%%%%%%%%
%%%%%%%%%%%%%%%%%%%%%%%%%%%%%%%%%%%%%
\section{Reference Implementation}
\label{sec:software_design:ref_impl}

%
\subsection{ID generation}
    To generate a unique ID for a given device we are combining all available MAC addresses and combine them in one colon-free string.  
    As we go for openssl as a dependency for PBKDF2 we are using it's SHA512 digest function to generate strong hashes.  
     sha512
     pbkdf2
     aes256
     mtd 
     
     We further enhance the brute forcing protection in reducing the generated hash to the first 32 Byte, which leaves around $1.158e^{77}$ possible combinations. 
     As shown in equation \ref{TODO} this increases the risk of a collision, but the risk is still very low. If an collision happens, it is handled on the server side.
%%%%%%%%%%%%%%%%%%%%%%%%%%%%%%%%%%%%%
%%%%%%%%%%%%%%%%%%%%%%%%%%%%%%%%%%%%%
%%%%%%%%%%%%   SECTION   %%%%%%%%%%%%
%%%%%%%%%%%%%%%%%%%%%%%%%%%%%%%%%%%%%
%%%%%%%%%%%%%%%%%%%%%%%%%%%%%%%%%%%%%
\section{Summary}

In this chapter, we gave an introduction to the data collection application (\ref{sec:software_design:data_collection} and described the concept of data transmission in (section \ref{sec:software_design:tx}).
We also introduced the reference implementation for the proposed solution in section \ref{sec:software_design:ref_impl}. 
%
\chapter{Collection Server}
\label{chap:mmeasurement}
To collect only dates, that are relevant for the purpose of the survey and it's safe transmission, the client has a mayor role. The server however is also important for keeping the data safe and consistent. Without a correctly handled data processing the best collection software is meaningless if it can be gathered by the developer.\\
The timely decryption and decoding of the incoming data is of same importance as the correct storage solution for the amount of expected information. It is also important to know, what can't be done with the server.

While we discussed the client side in the last chapter, this chapter is going to focus on
the server side implementation of the data collection process.\\
We are giving some general recommendations again in the first section and talk more in depth about our implementation in section \ref{sec:measurement:eval_setup}.
In section \ref{sec:measurement:robust} we are discussing issues, that couldn't be solved with our implementation or with our concept in general.
%


%%%%%%%%%%%%%%%%%%%%%%%%%%%%%%%%%%%%%
%%%%%%%%%%%%%%%%%%%%%%%%%%%%%%%%%%%%%
%%%%%%%%%%%%   SECTION   %%%%%%%%%%%%
%%%%%%%%%%%%%%%%%%%%%%%%%%%%%%%%%%%%%
%%%%%%%%%%%%%%%%%%%%%%%%%%%%%%%%%%%%%
\section{Measurement Limitations}
\label{sec:measurement:limits}
%
If no additional PII was collected, the only information, which could be used for reidentification is the ID. 
Nevertheless the data should be stored in secure way. It should not be accessible from outside your network setup to avoid data leakage. 
%

%%%%%%%%%%%%%%%%%%%%%%%%%%%%%%%%%%%%%
%%%%%%%%%%%%%%%%%%%%%%%%%%%%%%%%%%%%%
%%%%%%%%%%%%   SECTION   %%%%%%%%%%%%
%%%%%%%%%%%%%%%%%%%%%%%%%%%%%%%%%%%%%
%%%%%%%%%%%%%%%%%%%%%%%%%%%%%%%%%%%%%
\section{Evaluation Setup}
\label{sec:measurement:eval_setup}
%


%
\newpage
%
 

%%%%%%%%%%%%%%%%%%%%%%%%%%%%%%%%%%%%%
%%%%%%%%%%%%%%%%%%%%%%%%%%%%%%%%%%%%%
%%%%%%%%%%%%   SECTION   %%%%%%%%%%%%
%%%%%%%%%%%%%%%%%%%%%%%%%%%%%%%%%%%%%
%%%%%%%%%%%%%%%%%%%%%%%%%%%%%%%%%%%%%
\section{Data Robustness}
\label{sec:measurement:robust}
%


%
\newpage
%
 

%%%%%%%%%%%%%%%%%%%%%%%%%%%%%%%%%%%%%
%%%%%%%%%%%%%%%%%%%%%%%%%%%%%%%%%%%%%
%%%%%%%%%%%%   SECTION   %%%%%%%%%%%%
%%%%%%%%%%%%%%%%%%%%%%%%%%%%%%%%%%%%%
%%%%%%%%%%%%%%%%%%%%%%%%%%%%%%%%%%%%%
\section{Summary}


%

\chapter{Results}
\label{chap:results}
In this chapter we provide information of our evaluation steps and discuss the results of our measurement setup described in .\\


%%%%%%%%%%%%%%%%%%%%%%%%%%%%%%%%%%%%%
%%%%%%%%%%%%%%%%%%%%%%%%%%%%%%%%%%%%%
%%%%%%%%%%%%   SECTION   %%%%%%%%%%%%
%%%%%%%%%%%%%%%%%%%%%%%%%%%%%%%%%%%%%
%%%%%%%%%%%%%%%%%%%%%%%%%%%%%%%%%%%%%
\section{Anonymization Results}
\label{sec:results:anon}
%




%%%%%%%%%%%%%%%%%%%%%%%%%%%%%%%%%%%%%
%%%%%%%%%%%%%%%%%%%%%%%%%%%%%%%%%%%%%
%%%%%%%%%%%%   SECTION   %%%%%%%%%%%%
%%%%%%%%%%%%%%%%%%%%%%%%%%%%%%%%%%%%%
%%%%%%%%%%%%%%%%%%%%%%%%%%%%%%%%%%%%%
\section{Telemetry statistics}
\label{sec:results:telemetry}
%




%%%%%%%%%%%%%%%%%%%%%%%%%%%%%%%%%%%%%
%%%%%%%%%%%%%%%%%%%%%%%%%%%%%%%%%%%%%
%%%%%%%%%%%%   SECTION   %%%%%%%%%%%%
%%%%%%%%%%%%%%%%%%%%%%%%%%%%%%%%%%%%%
%%%%%%%%%%%%%%%%%%%%%%%%%%%%%%%%%%%%%
\section{Summary}
As expected, the user systems are fully anonymized ... and several different devices reported to our servers

\chapter{Conclusion and Outlook}
\label{chap:conclusion}

%%%%%%%%%%%%%%%%%%%%%%%%%%%%%%%%%%%%%
%%%%%%%%%%%%%%%%%%%%%%%%%%%%%%%%%%%%%
%%%%%%%%%%%%   SECTION   %%%%%%%%%%%%
%%%%%%%%%%%%%%%%%%%%%%%%%%%%%%%%%%%%%
%%%%%%%%%%%%%%%%%%%%%%%%%%%%%%%%%%%%%
\section{Summary}
In this thesis we provided a design for data collection and transmission, as well as a reference implementation for the client- and server-side. We have demonstrated a system to transmit data, detached from the system's network interface information to provide anonymity of sender. 
To realize this, we utilize the domain name system with its hierarchical structure. Only a interposed DNS server communicates with the collection server, therefore, no client IP can be stored.\\


%


%%%%%%%%%%%%%%%%%%%%%%%%%%%%%%%%%%%%%
%%%%%%%%%%%%%%%%%%%%%%%%%%%%%%%%%%%%%
%%%%%%%%%%%%   SECTION   %%%%%%%%%%%%
%%%%%%%%%%%%%%%%%%%%%%%%%%%%%%%%%%%%%
%%%%%%%%%%%%%%%%%%%%%%%%%%%%%%%%%%%%%
\section{Future Directions}

While we have shown significant gains, over some existing systems, we couldn't solve the problem of data validation. While systems with user authentication can use these to increase the workload needed to create counterfeit data it may reduce participation distinctly.
A good solution is needed here and requires additional research.



%BIBLIOGRAPHY
\addcontentsline{toc}{chapter}{Bibliography}
\bibliographystyle{IEEEtran}
\fancyhead[RE]{\normalfont Bibliography}
\fancyhead[LO]{\normalfont Bibliography}
\bibliography{bibliography/IEEEabrv,bibliography/example}


%APPENDIX
\pagestyle{plain}
\appendix
\chapter{Appendix}
\label{chap:appendix}

%%% TODO: Put a caption to it
\begin{enumerate}
    \label{list:brave_question}
    \item How long has this browser been open for the last seven days?
    \item Have you made Brave your default browser?
    \item Did you follow the on-boarding process
    \item Did you import settings and bookmarks, and if so from where?
    \item How many bookmarks do you have?
    \item How many open windows do you have?
    \item How many open tabs do you have?
    \item What has been your interaction with the shields icon?
    \item Have you ever used a private window?
    \item Have you ever used a Tor private window?
    \item What is the state of your Brave Rewards wallet? [NOT IMPLEMENTED]
    \item How much BAT, excluding grants, is in your wallet?
    \item  Have you made use of Auto-contribute in Brave Rewards?
    \item Have you made use of tips within Brave Rewards?
    \item Have you enabled sync?
    \item How many extensions have you installed?
    \item Have you enabled Brave Ads?
    \item How many questions the browser was able to answer within a week?
    \item How many times did you search last week?
    \item Which is your currently selected search engine?
    \item How much data did Brave save you last week?
    \item Is crash reporting enabled?
    \item Have you used SpeedReader?
    \item Number of times user clicked on SpeedReader button to toggle the feature?
    \item On average, how many New Tab Pages did you create per day?
    \item Is the sponsored new tab page option enabled?
    \item On average, how many of the New Tab Pages you saw per day were sponsored?
    \item On the New Tab Page, did you click the Customize Settings icon?
\end{enumerate}


\backmatter
%\include{cv}

\end{document}
